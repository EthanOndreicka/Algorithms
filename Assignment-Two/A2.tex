%%%%%%%%%%%%%%%%%%%%%%%%%%%%%%%%%%%%%%%%%
%
% CMPT xxx
% Some Semester
% Lab/Assignment/Project X
%
%%%%%%%%%%%%%%%%%%%%%%%%%%%%%%%%%%%%%%%%%

%%%%%%%%%%%%%%%%%%%%%%%%%%%%%%%%%%%%%%%%%
% Short Sectioned Assignment
% LaTeX Template
% Version 1.0 (5/5/12)
%
% This template has been downloaded from: http://www.LaTeXTemplates.com
% Original author: % Frits Wenneker (http://www.howtotex.com)
% License: CC BY-NC-SA 3.0 (http://creativecommons.org/licenses/by-nc-sa/3.0/)
% Modified by Alan G. Labouseur  - alan@labouseur.com
%
%%%%%%%%%%%%%%%%%%%%%%%%%%%%%%%%%%%%%%%%%

%----------------------------------------------------------------------------------------
%	PACKAGES AND OTHER DOCUMENT CONFIGURATIONS
%----------------------------------------------------------------------------------------

\documentclass[letterpaper, 10pt,DIV=13]{scrartcl} 

\usepackage[T1]{fontenc} % Use 8-bit encoding that has 256 glyphs
\usepackage[english]{babel} % English language/hyphenation
\usepackage{amsmath,amsfonts,amsthm,xfrac} % Math packages
\usepackage{sectsty} % Allows customizing section commands
\usepackage{graphicx}
\usepackage[lined,linesnumbered,commentsnumbered]{algorithm2e}
\usepackage{listings}
\usepackage{parskip}
\usepackage{lastpage}
\usepackage{tabularx}

\allsectionsfont{\normalfont\scshape} % Make all section titles in default font and small caps.

\usepackage{fancyhdr} % Custom headers and footers
\pagestyle{fancyplain} % Makes all pages in the document conform to the custom headers and footers

\fancyhead{} % No page header - if you want one, create it in the same way as the footers below
\fancyfoot[L]{} % Empty left footer
\fancyfoot[C]{} % Empty center footer
\fancyfoot[R]{page \thepage\ of \pageref{LastPage}} % Page numbering for right footer

\renewcommand{\headrulewidth}{0pt} % Remove header underlines
\renewcommand{\footrulewidth}{0pt} % Remove footer underlines
\setlength{\headheight}{13.6pt} % Customize the height of the header

\numberwithin{equation}{section} % Number equations within sections (i.e. 1.1, 1.2, 2.1, 2.2 instead of 1, 2, 3, 4)
\numberwithin{figure}{section} % Number figures within sections (i.e. 1.1, 1.2, 2.1, 2.2 instead of 1, 2, 3, 4)
\numberwithin{table}{section} % Number tables within sections (i.e. 1.1, 1.2, 2.1, 2.2 instead of 1, 2, 3, 4)

\setlength\parindent{0pt} % Removes all indentation from paragraphs.

\binoppenalty=3000
\relpenalty=3000

\usepackage{xcolor}

\definecolor{codeblue}{rgb}{0,0.4,0.9}
\definecolor{codegray}{rgb}{0.5,0.5,0.5}
\definecolor{codepurple}{rgb}{0.58,0,0.82}
\definecolor{backcolour}{rgb}{0.95,0.95,0.92}

\lstdefinestyle{mystyle}{
    backgroundcolor=\color{backcolour},   
    commentstyle=\color{codeblue},
    keywordstyle=\color{red},
    numberstyle=\tiny\color{codegray},
    stringstyle=\color{codepurple},
    basicstyle=\ttfamily\footnotesize,
    breakatwhitespace=false,         
    breaklines=true,                 
    captionpos=b,                    
    keepspaces=true,                 
    numbers=left,                    
    numbersep=5pt,                  
    showspaces=false,                
    showstringspaces=false,
    showtabs=false,                  
    tabsize=3
}

\lstset{style=mystyle}

%----------------------------------------------------------------------------------------
%	TITLE SECTION
%----------------------------------------------------------------------------------------

\newcommand{\horrule}[1]{\rule{\linewidth}{#1}} % Create horizontal rule command with 1 argument of height

\title{	
   \normalfont \normalsize 
   \textsc{CMPT 435 - Fall 2023 - Dr. Labouseur} \\[10pt] % Header stuff.
   \horrule{0.5pt} \\[0.25cm] 	% Top horizontal rule
   \huge Assignment Two  \\     	    % Assignment title
   \horrule{0.5pt} \\[0.25cm] 	% Bottom horizontal rule
}

\author{Ethan Ondreicka \\ \normalsize Ethan.Ondreicka1@Marist.edu}

\date{\normalsize\today} 	% Today's date.

\begin{document}
\maketitle % Print the title

%----------------------------------------------------------------------------------------
%   start Part One
%----------------------------------------------------------------------------------------
\section{Part One: Linear and Binary Search}

\subsection{Developing the Linear Search Function}

\begin{lstlisting}[language=c++, caption= Construction of the Linear Search Function]
// Linear Search Function
bool linearSearch(const std::string target, const std::string lines[], int lineCount, int& comparisons) {
    comparisons = 0; // counter for comparisons
    for (int i = 0; i < lineCount; i++) {
        comparisons++; // Increment the comparisons count for each iteration
        if (lines[i] == target) {
            return true; // Item found
        }
    }
    return false; // Item not found
}
\end{lstlisting}
The linear search function takes the list of 666 magic items and sequentially goes through each one until the target value is found. As you can see on in the for loop starting on line 4, it will loop for as long as it takes to find the target value. The asymptotic running time of this function is $O(n)$.
\subsection{Developing the Binary Search Function}
\begin{lstlisting}[language=c++, caption= Creation of the Binary Search Function]
// Binary Search Function
bool binarySearch(const std::string target, const std::string lines[], int lineCount, int& comparisons) {
    comparisons = 0; // comparisons counter
    int leftValue = 0;
    int rightValue = lineCount - 1;
    while (leftValue <= rightValue) {
        comparisons++; // count for each iteration
        int middleValue = leftValue + (rightValue - leftValue) / 2;
        if (lines[middleValue] == target) {
            return true; // Item found
        }
        if (lines[middleValue] < target) {
            leftValue = middleValue + 1;
        } else {
            rightValue = middleValue - 1;
        }
    }
    return false; // Item not found
}
\end{lstlisting}
The Binary Search function takes the magic items list and will grab the middle value. If the target is not found, it will divide the list in half and grab the middle value again. It will do it until the middle value is the target value (as seen on line 8). If the target is found, the function will return true. The asymptotic running time of this function is $O(logn)$

\section{Part Two: Hashing}
\subsection{Creation of the Hash Table}
\begin{lstlisting}[language=c++, caption= Hashing Table Class]
// Class for the hash table itself
class HashTable {
public:
    HashTable() {
        table.resize(TABLE_SIZE);
    }

    void insert(const std::string& key, const std::string& value) {
        unsigned int index = hash(key);
        Node* newNode = new Node(key);
        newNode->setNext(table[index]); // sets next pointer to current head
        table[index] = newNode;
    }

    bool get(const std::string& key, std::string& value, int& comparisons) {
        unsigned int index = hash(key);
        comparisons = 0;
        Node* current = table[index]; // starts at head
        while (current != nullptr) {
            comparisons++;
            if (current->getItem() == key) {
                value = current->getItem(); 
                return true;      // key is found
            }
            current = current->getNext();
        }
        return false;   // key not found
    }
    std::vector<Node*> table; // linked lists to store table
};
\end{lstlisting}
The Hashing Class takes all of the magic items and stores them into a hash table of size 250. The get function is used to retrieve and count comparisons while searching for the target value (key). The asymptotic running time for hashing is $O(n)$ while the asymptotic running time of the chaining is $O(\alpha + 1)$. The reason chaining is $O(\alpha + 1)$ is due to the load factor.
\section{Part Three: Results}
\subsection{Results Table}
\centering
\begin{tabularx}{0.8\textwidth} { 
  | >{\raggedright\arraybackslash}X 
  | >{\centering\arraybackslash}X 
  | >{\raggedleft\arraybackslash}X | }
 \hline
 Search Function & Average Comparisons & Asymptotic Running Time\\
 \hline
 Linear Search  & 319.93 & $O(n)$  \\
 \hline
 Binary Search  & 9.64  & $O(logn)$  \\
 \hline
 Hashing and Chaining & 2.31  & $O(n)$ and $O(\alpha + 1)$ \\
\hline
\end{tabularx}


\footnotesize{The Average Comparisons is based on each function running 42 times for the 42 random items}
\end{document}