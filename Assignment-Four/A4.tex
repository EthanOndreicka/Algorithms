%%%%%%%%%%%%%%%%%%%%%%%%%%%%%%%%%%%%%%%%%
%
% CMPT xxx
% Some Semester
% Lab/Assignment/Project X
%
%%%%%%%%%%%%%%%%%%%%%%%%%%%%%%%%%%%%%%%%%

%%%%%%%%%%%%%%%%%%%%%%%%%%%%%%%%%%%%%%%%%
% Short Sectioned Assignment
% LaTeX Template
% Version 1.0 (5/5/12)
%
% This template has been downloaded from: http://www.LaTeXTemplates.com
% Original author: % Frits Wenneker (http://www.howtotex.com)
% License: CC BY-NC-SA 3.0 (http://creativecommons.org/licenses/by-nc-sa/3.0/)
% Modified by Alan G. Labouseur  - alan@labouseur.com
%
%%%%%%%%%%%%%%%%%%%%%%%%%%%%%%%%%%%%%%%%%

%----------------------------------------------------------------------------------------
%	PACKAGES AND OTHER DOCUMENT CONFIGURATIONS
%----------------------------------------------------------------------------------------

\documentclass[letterpaper, 10pt,DIV=13]{scrartcl} 

\usepackage[T1]{fontenc} % Use 8-bit encoding that has 256 glyphs
\usepackage[english]{babel} % English language/hyphenation
\usepackage{amsmath,amsfonts,amsthm,xfrac} % Math packages
\usepackage{sectsty} % Allows customizing section commands
\usepackage{graphicx}
\usepackage[lined,linesnumbered,commentsnumbered]{algorithm2e}
\usepackage{listings}
\usepackage{parskip}
\usepackage{lastpage}
\usepackage{tabularx}

\allsectionsfont{\normalfont\scshape} % Make all section titles in default font and small caps.

\usepackage{fancyhdr} % Custom headers and footers
\pagestyle{fancyplain} % Makes all pages in the document conform to the custom headers and footers

\fancyhead{} % No page header - if you want one, create it in the same way as the footers below
\fancyfoot[L]{} % Empty left footer
\fancyfoot[C]{} % Empty center footer
\fancyfoot[R]{page \thepage\ of \pageref{LastPage}} % Page numbering for right footer

\renewcommand{\headrulewidth}{0pt} % Remove header underlines
\renewcommand{\footrulewidth}{0pt} % Remove footer underlines
\setlength{\headheight}{13.6pt} % Customize the height of the header

\numberwithin{equation}{section} % Number equations within sections (i.e. 1.1, 1.2, 2.1, 2.2 instead of 1, 2, 3, 4)
\numberwithin{figure}{section} % Number figures within sections (i.e. 1.1, 1.2, 2.1, 2.2 instead of 1, 2, 3, 4)
\numberwithin{table}{section} % Number tables within sections (i.e. 1.1, 1.2, 2.1, 2.2 instead of 1, 2, 3, 4)

\setlength\parindent{0pt} % Removes all indentation from paragraphs.

\binoppenalty=3000
\relpenalty=3000

\usepackage{xcolor}

\definecolor{codeblue}{rgb}{0,0.4,0.9}
\definecolor{codegray}{rgb}{0.5,0.5,0.5}
\definecolor{codepurple}{rgb}{0.58,0,0.82}
\definecolor{backcolour}{rgb}{0.95,0.95,0.92}

\lstdefinestyle{mystyle}{
    backgroundcolor=\color{backcolour},   
    commentstyle=\color{codeblue},
    keywordstyle=\color{red},
    numberstyle=\tiny\color{codegray},
    stringstyle=\color{codepurple},
    basicstyle=\ttfamily\footnotesize,
    breakatwhitespace=false,         
    breaklines=true,                 
    captionpos=b,                    
    keepspaces=true,                 
    numbers=left,                    
    numbersep=5pt,                  
    showspaces=false,                
    showstringspaces=false,
    showtabs=false,                  
    tabsize=3
}

\lstset{style=mystyle}

%----------------------------------------------------------------------------------------
%	TITLE SECTION
%----------------------------------------------------------------------------------------

\newcommand{\horrule}[1]{\rule{\linewidth}{#1}} % Create horizontal rule command with 1 argument of height

\title{	
   \normalfont \normalsize 
   \textsc{CMPT 435 - Fall 2023 - Dr. Labouseur} \\[10pt] % Header stuff.
   \horrule{0.5pt} \\[0.25cm] 	% Top horizontal rule
   \huge Assignment Four - The Finale  \\     	    % Assignment title
   \horrule{0.5pt} \\[0.25cm] 	% Bottom horizontal rule
}

\author{Ethan Ondreicka \\ \normalsize Ethan.Ondreicka1@Marist.edu}

\date{\normalsize\today} 	% Today's date.

\begin{document}
\maketitle % Print the title

%----------------------------------------------------------------------------------------
%   start Part One
%----------------------------------------------------------------------------------------
\section{Part One: Implementing	the	Bellman-Ford dynamic programming algorithm for Single Source Shortest	Path}

\subsection{Parsing the File}

\begin{lstlisting}[language=c++, caption= Parsing through the input file]
// file reader for the graphs2.txt
void parseInputFile(const std::string& filename, Graph& graph) {
    std::ifstream inputFile(filename);
    if (!inputFile.is_open()) {
        std::cerr << "Error opening file." << std::endl;
        return;
    }

    std::string line;
    bool newGraph = true; // Flag to indicate a new graph
    while (std::getline(inputFile, line)) {
        if (line.empty() || line.substr(0, 2) == "--") {
            newGraph = true; // Indicates a new graph
            continue;
        }

        if (newGraph) {
            graph.clearVertices(); // Clear existing graph data for the new graph
            newGraph = false; // Reset flag
        }

        if (line.substr(0, 3) == "add") {
            std::istringstream iss(line);
            std::string command;
            iss >> command;

            if (command == "add") {
                std::string subcommand;
                iss >> subcommand;

                if (subcommand == "vertex") {
                    int vertexId;
                    iss >> vertexId;
                    graph.addVertex(vertexId);
                } else if (subcommand == "edge") {
                    int src, dest, weight;
                    char dash;
                    iss >> src >> dash >> dest >> weight;
                    graph.addEdge(src, dest, weight);
                }
            }
        }
    }

    inputFile.close();
}
\end{lstlisting}
The function parses through a given input file to create and clear graphs.
\subsection{Vertex Constructor}
\begin{lstlisting}[language=c++, caption= Constructing the Vertex Object]
// contructs each vertex
struct Vertex {
    int id;
    bool processed;
    std::vector<Vertex*> neighbors;
    std::vector<int> weights;
    long long distance; //  store distance for Bellman-Ford
    Vertex* predecessor; // store predecessor for path reconstruction
    Vertex(int _id) : id(_id), processed(false), distance(std::numeric_limits<int>::max()), predecessor(nullptr) {}
};
\end{lstlisting}
This creates each new vertex as well as assigns the neighbors (edges) and their weights

\subsection{The Bellman Ford Algorithm}

\begin{lstlisting}[language=c++, caption= Bellman-Ford Algorithm]
    void bellmanFord(Vertex* source, int maxIterations) {
        initializeSingleSource(source);

        for (size_t i = 0; i < maxIterations; ++i) {
            for (std::unordered_map<int, Vertex*>::iterator it = vertices.begin(); it != vertices.end(); ++it) {
                Vertex* u = it->second;
                for (size_t j = 0; j < u->neighbors.size(); ++j) {
                    relax(u, u->neighbors[j], u->weights[j]);
                }
            }
        }
    }
\end{lstlisting}
This function  loops through all of the vertices in the graph specified by 'maxIterations' and attempts to relax the distance. This operation is $O(n)$.
\newpage
\section{Part Two: The Fractional Knapsack}
\subsection{Constructors}
\begin{lstlisting}[language=c++, caption= Spice and Knapsack Constructor]
// constructs the knapsack object 
struct Knapsack {
    int knapsackNumber;
    int capacity;
    double totalValue;
};

// construction of the spice object 
struct Spice {
    std::string name;
    double total_price;
    int qty;
    double price_per_unit;
};
\end{lstlisting}
These constructors make the knapsack object as well as the spice object. The values for each of the attributes in the constructor are based on the input file.
\subsection{Sorting}
\begin{lstlisting}[language=c++, caption= Insertion Sort Algorithm]
// Insertion Sort Function for spices based on price per unit in descending order
void insertionSort(std::vector<Spice>& spices) {
    for (size_t i = 1; i < spices.size(); ++i) {
        Spice valueToInsert = spices[i];
        int j = i - 1;
        while (j >= 0 && spices[j].price_per_unit < valueToInsert.price_per_unit) {
            spices[j + 1] = spices[j];
            j = j - 1;
        }
        spices[j + 1] = valueToInsert;
    }
}
\end{lstlisting}
I chose to use the insertion sort algorithm to sort the spice prices (in descending order) before they fill the knapsacks. Because I decided to use the Insertion Sort Algorithm, the asymptotic running time for the partial knapsack problem is $O(n^2)$.
\pagebreak

\subsection{Filling the Knapsack!}
\begin{lstlisting}[language=c++, caption= Code to fill the knapsack with spices]
int totalQuantity = 0; // Total available quantity of spices

    // Calculate the total quantity of available spices
    for (size_t i = 0; i < spices.size(); ++i) {
        const Spice& spice = spices[i];
        totalQuantity += spice.qty;
    }

    for (size_t i = 0; i < knapsacks.size(); ++i) {
        Knapsack& knapsack = knapsacks[i];
        std::cout << "Knapsack of capacity " << knapsack.capacity << " is worth ";

        int remainingCapacity = knapsack.capacity;
        std::vector<std::string> contents;
        std::vector<Spice> availableSpices = spices; // Create a copy for this knapsack

        // Loop through le spices and fill the current knapsack
        for (size_t j = 0; j < availableSpices.size(); ++j) {
            Spice& spice = availableSpices[j];

            while (remainingCapacity > 0 && spice.qty > 0) {
                if (spice.qty <= remainingCapacity) {
                    contents.push_back(std::to_string(spice.qty) + " scoops of " + spice.name);
                    remainingCapacity -= spice.qty;
                    knapsack.totalValue += spice.total_price;
                    spice.qty = 0; // All scoops used for this knapsack
                } else {
                    contents.push_back(std::to_string(remainingCapacity) + " scoops of " + spice.name);
                    spice.qty -= remainingCapacity;
                    knapsack.totalValue += remainingCapacity * spice.price_per_unit;
                    remainingCapacity = 0; // Knapsack filled to capacity for this knapsack
                }
            }
        }
}
\end{lstlisting}
This first for loop from lines 4-7 show the calculations to see how much spices there to fill your knapsack. After that, the bigger for loop from lines 9-35 actually fill the knapsack using a greedy solution. 
\end{document}